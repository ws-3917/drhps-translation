\documentclass[UTF8, a4paper, 12pt]{ctexart}
\usepackage{amsmath, amsthm, amssymb, appendix, bm, graphicx, hyperref, mathrsfs}
\hypersetup{
    colorlinks=true,
    linkcolor=black
}
\title{\textbf{雨伞架汉化组工作实施办法(试行)}}
\author{雨伞架组内成员\ 一同}
\date{\today}
\linespread{1.5}
\begin{document}
\setcounter{page}{0}
\thispagestyle{empty}
\maketitle
\section*{前言}
雨伞架汉化组工作实施办法(以下简称《办法》)旨在规范雨伞架汉化组工作的流程,明确现阶段工作的重点和任务,提高汉化工作和其他任务的效率和质量。为雨伞架汉化组全体成员提供明确的指导,确保各项工作的顺利进行。

《办法》包含\emph{七个部分},分别是:

\begin{enumerate}
    \item \nameref{sec1}
    \item \nameref{sec2}
    \item \nameref{sec3}
    \item \nameref{sec4}
    \item \nameref{sec5}
    \item \nameref{sec6}
    \item \nameref{sec7}
\end{enumerate}

《办法》适用于雨伞架汉化组全体成员,自发布之日起实施。
\newpage
\setcounter{page}{1}
\tableofcontents
\newpage
\section{重点任务、决策与目标}\label{sec1}
\subsection{重点任务与目标}
本部分所述“任务”指 2025-03-02 至 2025-03-14 期间汉化组的重点工作方向,后续任务安排将视完成情况动态调整。

\subsubsection{PS:OT 维基百科建设}
\paragraph{目标} 规划维基百科顶层框架,完成\emph{首页、顶层分类}的简介页面,并为每一分类编写一篇\emph{模板页}。

\paragraph{任务} 工作重心集中在框架的搭建上,后续将根据维基百科的实际情况,逐步完善每一分类的子页面内容。

人员分工,格式规范,具体内容的编写指导等,参见第 \ref{sec3} 节。

\subsubsection{PS:OT 边框与贴图上色}
\paragraph{目标} 至少完成 1 个边框,贴图上色进度完成 70\%以上。

\paragraph{任务} 由 \nameref{xiaoyun} 集中完成剩余的游戏边框设计,其他负责美术的成员专注于完成上色版贴图的制作。

\subsubsection{PS:OT 文本校对}
\paragraph{目标} 完成 Outlands, Starton, Foundry 三个场景的文本格式校对工作,并初步完成上述三个场景的内容校对工作。

\paragraph{任务} 由 \nameref{mdr}, \nameref{braing}, \nameref{errosia} 完成,具体工作、流程参见第 \ref{sec5} 节。

\subsubsection{DR:HPS UI贴图汉化}
\paragraph{目标} 完成 DR:HPS \textbf{至少80\%} 非文本贴图的汉化工作,此类贴图包括:

\begin{itemize}
    \item 战斗按钮
    \item 主场景标牌、建筑文字
    \item UI贴图
\end{itemize}

\subsubsection{DR:HPS 长文本汉化}
\paragraph{目标} 完成 DR:HPS \textbf{至少80\%} 长文本的汉化工作,此类文本包括:

\begin{itemize}
    \item 信件
    \item 报道
    \item 研究记录
    \item 便签
    \item “猜想”介绍
\end{itemize}

\paragraph{任务} 长文本包括贴图内的文本和直接提取的长文本,现阶段无须绘制对应贴图,只需翻译内容。

本项任务由 \nameref{baisha},\nameref{aryen},\nameref{zhazha},\nameref{difena},\nameref{zhiyuan} 完成。

具体分工参见第 \ref{sec4} 节。

\subsubsection{DR:HPS 程序工作}
\paragraph{目标} 在 2025-03-12 之前(最晚不晚于 03-14)将已翻译文本及UI贴图全部导入游戏,并发布初版构建。

\paragraph{任务} 由 \nameref{ws3917} 解决打字机速率、中文字体显示与间距问题,并探索自动化导入文本及贴图的方法。

本任务可考虑邀请其他人协助字体制作/合成。

\subsubsection{DR:HPS 文本翻译}

\paragraph{目标} 完成 DR:HPS \textbf{至少80\%} 短文本\emph{(长度小于 30 词的标题/对话)}的初步翻译工作。

\paragraph{任务} 翻译工作由 \nameref{wasneet},\nameref{ax} 和 \nameref{krisdm} 完成,对于无法敲定的文本可先跳过。若提前完成,可启动短文本校对。

具体工作流程和任务分配,参见第 \ref{sec4} 节。

\subsection{重要决策}
\subsubsection{2025-04-02 起,不再继续维护 PS:OT 的本体汉化工作。}
此后,将不会对 PS:OT 的\emph{文本、贴图、程序}进行任何调整。包括:
\begin{itemize}
    \item 校对工作
    \item 新功能/特性
    \item Bug修复
\end{itemize}

同时,也不会使用“PS域外传说汉化组”的B站账号继续发布域外的汉化预览、流程展示等内容。

但会继续维护 PS:OT 的维基百科页面。

\subsubsection{DR:HPS只发布简体中文-人名不翻译单版本}

出于时间成本与效率的考虑,放弃DR:HPS汉化项目\emph{繁体中文}与\emph{人名翻译}的制作,最终发布版本为Windows PC版。

安卓版移植情况待定,视程序难度决定是否制作。

\subsubsection{注册 DR:HPS 汉化组B站账号}

2025-03-10 前,由 WS3917 注册账号,并将账号登录、管理的权限交给不少于\emph{3名} DR:HPS 汉化成员。

账号的具体管理与推广工作详见第 \ref{sec6} 节。

\newpage
\section{任务分配、验收与调整}\label{sec2}

\newcommand{\newtask}[6]{
    \subsubsection{#1}
    \begin{itemize}
        \item \textbf{任务}\  #2
        \item \textbf{DDL}\  #3
        \item \textbf{备注}\  #4
        \item \textbf{已确认?}\  #5
        \item \textbf{已完成?}\  #6
    \end{itemize}
}
\subsection{总则}
\paragraph{DDL} 意为\textbf{“截止日期”},指任务应该完成的时间节点。设置DDL 的目的是为了确保任务按时完成,避免拖延。

DDL的设置由 \nameref{ws3917} 进行,由被分配者本人确认可以胜任后生效,临近 DDL 时会进行组内相互提醒。如果临时出现时间安排变动或突发情况,请联系 \nameref{ws3917} 进行相应调整。

\paragraph{任务量} 根据成员的空闲时间多少,设定不同的任务量与 DDL。一般来说,时间越紧张/不固定,DDL会越宽松,任务量也会越少。

无特殊情况下,任务分配周期最短为 1 天,通常为半周(3-4 天),最长不会超过 1 周。

\paragraph{任务目标} 大多数任务的进度(文本翻译、贴图上色、贴图汉化)可量化,但部分任务(如程序构建、页面校对)无法量化。因此,任务目标主要分为两类:
\begin{itemize}
    \item \textbf{具体指标} :如“翻译 \verb|0-Headers #235 - #280| 的短文本” 或 “上色 \verb|Alphys| 文件夹内全部32张头像贴图”;
    \item \textbf{模糊目标} :如“探索快捷导入文本的方法” 或 “优化 PS:OT Wiki的 ‘奶油糖果派’页面框架设计”。
\end{itemize}

任务\emph{完成}后,请及时告知 \nameref{ws3917},以便对任务进行验收与调整。在次日安排新任务前,可自由选择休息,或者参与其他任务,协助其他成员。

\paragraph{}

\subsection{WS3917}\label{ws3917}
\paragraph{主要职务} DR:HPS 程序构建
\paragraph{DDL分配周期} 无
\newpage
\subsection{Hola amigo}\label{hola}
\paragraph{主要职务} PS:OT Wiki 页面校对/框架设计
\paragraph{DDL分配周期} 双周5-6
\newpage

\subsection{Wasneet Potato}\label{wasneet}
\paragraph{主要职务} DR:HPS 短文本翻译 + 术语表 + PS:OT Wiki 页面编写(框架)
\paragraph{DDL分配周期} 1-2,3-4,5-6,7

\newtask{2025-03-03}{翻译DR:HPS 6-Unity-TextMeshPro  -- 349 - 462}{2025-03-04 (周二) 23:59}{
    对于不确定的文本,写评论记录,任务完成后在群里集中讨论。}{是}{是}

\newtask{2025-03-04}
{翻译DR:HPS 6-Unity-TextMeshPro  -- 目标条目数:80 \~{} 120条}
{2025-03-05 (周三) 23:59}
{无}{否}{否}

\newpage
\subsection{Murder-Sans--MDR}\label{mdr}
\paragraph{主要职务} PS:OT文本校对 + Wiki 页面辅助编写
\paragraph{DDL分配周期} 1-4,5-6,7

\newtask{2025-03-03}{校对PS:OT Starton -- Papyrus电话(不含Undyne)全部,编号4180 - 5000}{2025-03-06 (周四) 23:59}{
    可以与 \nameref{braing} 合作校对,如果出现争议记录评论,并及时上报。}{是}{否}

\newpage
\subsection{AX暗星233}\label{ax}
\paragraph{主要职务} DR:HPS 短文本翻译
\paragraph{DDL分配周期} 1-4,5-6,7

\newtask{2025-03-03}{翻译DR:HPS 6-Unity-TextMeshPro  -- 605 - 667}{2025-03-06 (周四) 23:59}{
    对于不确定的文本,写评论记录,任务完成后在群里集中讨论。}{是}{否}

\newpage
\subsection{Braing}\label{braing}
\paragraph{主要职务} PS:OT文本校对
\paragraph{DDL分配周期} 1-4,5-6,7

\newtask{2025-03-03}{校对PS:OT Starton -- Papyrus电话(不含Undyne)全部,编号4180 - 5000}{2025-03-06 (周四) 23:59}{
    可以与 \nameref{mdr} 合作校对,如果出现争议记录评论,并及时上报。}{否}{否}

\newpage

\subsection{KrisDreemurr}\label{krisdm}
\paragraph{主要职务} DR:HPS 短文本翻译 + 术语表
\paragraph{DDL分配周期} 1-4,5-6,7

\newtask{2025-03-03}{翻译DR:HPS 6-Unity-TextMeshPro  -- 1076 -- 1160}{2025-03-06 (周四) 23:59}{
    对于不确定的文本,写评论记录,任务完成后在群里集中讨论。}{是}{有问题的文本较多,需要修改}

\newpage
\subsection{1个渣渣}\label{zhazha}
\paragraph{主要职务} DR:HPS 长文本翻译
\paragraph{DDL分配周期} 1-5,6-7
\newtask{2025-03-03}{翻译DR:HPS 4-Unity-m\_text\_BIOSText -- 28 -- 33}{2025-03-07 (周五) 23:59}{
    对于不确定的文本,写评论记录,\textbf{尽量不要依赖AI翻译}}{否}{否}

\newpage
\subsection{屑moons月亮菌}\label{moons}
\paragraph{主要职务} PS:OT贴图上色
\paragraph{DDL分配周期} 1-5,6-7
\newtask{2025-03-03}{完成怪物小孩战斗(battleCharacters/kidd),及starton所有小怪(杰瑞、绅鼠猫、太空帽、雪铁龙、小酷龙)的上色。}{2025-03-07 (周五) 23:59}{如果遇到了困难可以提出。}{否}{否}

\newpage
\subsection{Errosia}\label{errosia}
\paragraph{主要职务} PS:OT文本辅助校对
\paragraph{DDL分配周期} 无
\newtask{2025-03-03}{查看PS:OT Outlands已有翻译建议,并视情况进行确认和评论}{2025-03-09 (周日) 23:59}{无}{否}{否}

\newpage
\subsection{凝雨白沙}\label{baisha}
\paragraph{主要职务} DR:HPS 长文本翻译/校对 + PS:OT贴图上色(待定)
AXAX\paragraph{DDL分配周期} 1-4,5-6,7
\newtask{2025-03-03}{翻译DR:HPS 4-Unity-m\_text\_BIOSText -- 21,25}{2025-03-06 (周四) 23:59}{
对于不确定的文本,写评论记录,任务完成后在群里集中讨论。}{是}{\~{}50\%}

\newpage
\subsection{这是纸鸢}\label{zhiyuan}
\paragraph{主要职务} DR:HPS UI 贴图汉化 + PS:OT贴图上色
\paragraph{DDL分配周期} 1-5,6-7
\newtask{2025-03-03}{绘制 DR:HPS 的 battle\_buttonsheet.png, battleui\_text\_popups .png, CHARICON\_QUEEN\_*.png的汉化贴图}{2025-03-07 (周五) 23:59}{
    都是UI或按钮贴图,可直接参考Deltarune的贴图。}{是}{ACT按钮和菜单文字是否翻译待敲定,其余均完成}

\newtask{2025-03-04}{绘制 DR:HPS 的tape\_mainseries*.png汉化贴图}{2025-03-07 (周五) 23:59}{无}{是}{否}
\newpage
\subsection{幻-\_-风}\label{huafeng}
\paragraph{主要职务} 暂无
\paragraph{DDL分配周期} 无

\newpage
\subsection{阿莱恩Aryen}\label{aryen}
\paragraph{主要职务} DR:HPS 长文本翻译/校对
\paragraph{DDL分配周期} 1-5,6-7
\newtask{2025-03-03}{翻译DR:HPS 4-Unity-m\_text\_BIOSText -- 109,110}{2025-03-07 (周五) 23:59}{
    对于不确定的文本,写评论记录。}{否}{否}

\newpage
\subsection{小云爱喝二锅头}\label{xiaoyun}
\paragraph{主要职务} PS:OT 边框绘制
\paragraph{DDL分配周期} 1-7
\newtask{2025-03-03}{完成首塔 Citadel 的边框设计}{2025-03-09 (周日) 23:59}{加急,可以提高报酬。}{否}{否}

\newpage
\subsection{C-G-O-A-T}\label{cgoat}
\paragraph{主要职务} PS:OT 贴图上色 + PS:OT Wiki 页面辅助编写(待定)
\paragraph{DDL分配周期} 1-4,5-6,7
\newtask{2025-03-03}{完成怪物小孩头像(dialogueCharacters/kidd)上色,及Alphys所有头像上色。}{2025-03-06 (周四) 23:59}{如果遇到了困难可以提出。}{否}{否}

\newpage
\subsection{Difena}\label{difena}
\paragraph{主要职务} DR:HPS 长文本辅助校对
\paragraph{DDL分配周期} 1-5,6-7
\newtask{2025-03-03}{翻译DR:HPS 4-Unity-m\_text\_BIOSText -- 111 -- 119}{2025-03-07 (周五) 23:59}{
    对于不确定的文本,写评论记录。}{否}{否}

\newpage

\subsection{Krab-K}\label{k}
\paragraph{主要职务} 暂无
\paragraph{DDL分配周期} 无

\newpage
\subsection{鼠饼}\label{shubing}
\paragraph{主要职务} PS:OT 贴图上色(待定)
\paragraph{DDL分配周期} 无

\newpage
\subsection{KodLenss}\label{kodlenss}
\paragraph{主要职务} PS:OT 边框绘制(待定)
\paragraph{DDL分配周期} 无
\newtask{2025-03-03}{完成空境 Aerialis 的 Archive 风格边框。}{2025-03-09 (周日) 23:59}{加急,可以提高报酬。}{否}{否}

\newpage

\subsection{屑箫}\label{xiexiao}
\paragraph{主要职务} 暂无
\paragraph{DDL分配周期} 无

\newpage
\section{维基百科建设与注意事项}\label{sec3}
\newpage
\section{DR:HPS 汉化工作}\label{sec4}
\newpage
\section{PS:OT 本体维护}\label{sec5}
\newpage
\section{项目宣传与推广}\label{sec6}
\newpage
\section{其他事项、附则与补充}\label{sec7}
\newpage

\end{document}